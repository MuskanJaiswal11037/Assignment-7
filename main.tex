



\usepackage[utf8]{inputenc}
\documentclass{beamer}
\usetheme{CambridgeUS}
\usepackage{listings}
\usepackage{blkarray}
\usepackage{listings}
\usepackage{subcaption}
\usepackage{url}
\usepackage{tikz}
\usepackage{tkz-euclide} % loads  TikZ and tkz-base
%\usetkzobj{all}
\usetikzlibrary{calc,math}
\usepackage{float}
\renewcommand{\vec}[1]{\mathbf{#1}}
\usepackage[export]{adjustbox}
\usepackage[utf8]{inputenc}
\usepackage{amsmath}
\usepackage{amsfonts}
\usepackage{tikz}
\usepackage{hyperref}
\usepackage{bm}
\usetikzlibrary{automata, positioning}
\providecommand{\pr}[1]{\ensuremath{\Pr\left(#1\right)}}
\providecommand{\mbf}{\mathbf}
\providecommand{\qfunc}[1]{\ensuremath{Q\left(#1\right)}}
\providecommand{\sbrak}[1]{\ensuremath{{}\left[#1\right]}}
\providecommand{\lsbrak}[1]{\ensuremath{{}\left[#1\right.}}
\providecommand{\rsbrak}[1]{\ensuremath{{}\left.#1\right]}}
\providecommand{\brak}[1]{\ensuremath{\left(#1\right)}}
\providecommand{\lbrak}[1]{\ensuremath{\left(#1\right.}}
\providecommand{\rbrak}[1]{\ensuremath{\left.#1\right)}}
\providecommand{\cbrak}[1]{\ensuremath{\left\{#1\right\}}}
\providecommand{\lcbrak}[1]{\ensuremath{\left\{#1\right.}}
\providecommand{\rcbrak}[1]{\ensuremath{\left.#1\right\}}}
\providecommand{\abs}[1]{\vert#1\vert}

\newcounter{saveenumi}
\newcommand{\seti}{\setcounter{saveenumi}{\value{enumi}}}
\newcommand{\conti}{\setcounter{enumi}{\value{saveenumi}}}
\usepackage{amsmath}
\setbeamertemplate{caption}[numbered]{}


\title{\typedef{ASSIGNMENT-7}}       
\author{MUSKAN JAISWAL -cs21btech11037}
\date{May 2022}
\logo{\large \Latex{}}
\begin{document}

\begin{frame}{Outline}
  \tableofcontents
\end{frame}
\section{Abstract}
	\begin{frame}{Abstract}
		\begin{itemize}
			\item 	This document contains the explanation of question  9.12  of Papoulis Pillai Probability book of chapter sequence of random variables.
		\end{itemize}
	\end{frame}
	
\maketitle

\section{QUESTION:}
\begin{frame}{}
\begin{block}{}

Show that :- If x(t) is a process with zero mean and auto correlation $f(t_1)f(t_2) w(t_1-t_2) $, then the process $y(t)=\frac{x(t)}{f(t)}$ is WSS with auto correlation $w(t_1-t_2) $. If x(t) is white noise with auto correlation $q(t_1) \times \delta(t_1-t_2)$
then the process $z(t) = \frac{x(t)}{q(t)}$ is WSS white noise with auto correlation 
$\delta(t_1-t_2) $.

\end{block}
\end{frame}
\section{ANSWERS: }
\begin{frame}{$1^{st}$ part:-}
According to question , x(t) is a zero mean process. So, E(x(T))=0 and 
$ u_y=u_x\int_{-\infty}^\infty f(\alpha) d\alpha $\\

$y(t)=\frac{x(t)}{f(t)}\\
R_{yy}(t_1,t_2)= \frac{R_{xx}(t_1,T_2)}{f(t_1)f(t_2)}\\
=\frac{f(t_1)f(t_2)w(t_1-t_2)}{f(t_1)f(t_2)}\\
=w(t_1-t_2)$ 

\end{frame}
\begin{frame}{$2^{nd} $ part:-}
x(t) is a white noise process which means \\
$R_x(t_1-t_2)=F^{-1}(N_0/2)= {N_0 \delta(t_1-t_2)}/2\\
\delta(x)=\{    \infty \hspace{0.5cm} x=0 \\
                  0 \hspace{0.5cm}   x\not=0 \} $
                  
                 This shows that  $X(t_1)$ and $x(t_2) $ are correlated only when $t_1=
                 t_2$ and uncorrelated when $t_1  \not = t_2$.
                 Therefore, $R_{xx}(t_1,t_2)=0$ , when $t_1$ is not equal to $t_2 $\\
                $ R_{xx}(t_1,t_2)=q(t_1) \times \delta(t_1-t_2) , when t_1=t_2 $\\
                 Hence, we can  write $R_{xx}(t_1,t_2) $ as
                 $ (q(t_1))^{1/2} (q(t_2))^{1/2} \times \delta(t_1-t_2) \hspace{0.5cm} (q(t_1)=q(t_2) )\\
                  z(t)=\frac{x(t)}{q(t)}         \\
                  R_{zz}=\frac{q(t_1) q(t_2) \delta(t_1-t_2) }{q(t_1)q(t_2)}\\
                  R_{zz}=\delta(t_1-t_2) $\\
                  Hence,proved 
            

\end{frame}
\end{document}
